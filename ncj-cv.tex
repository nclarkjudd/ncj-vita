\documentclass{ncj-vita}
\usepackage{fontspec}
\setmainfont[Ligatures=TeX]{ETBembo}

\def\myauthor{Nick Judd}
\def\mytitle{Vita}
\def\mycopyright{\myauthor}
\def\mykeywords{}
\def\mybibliostyle{plain}
\def\mybibliocommand{}
\def\mysubtitle{}
\def\myaffiliation{The University of Chicago}
\def\myaddress{\href{http://sociology.uchicago.edu}{Department of Sociology} \\ 1126 \textsc{E.} 59\textsc{th} \textsc{St.} \\ Chicago, \textsc{IL} 60637}
\def\myemail{ncj@uchicago.edu}
\def\myweb{http://www.nickjudd.com}
\def\myfax{(773) 702-4849}
\def\mytwitter{@nclarkjudd}
\def\myversion{}
\def\myrevision{}

\input{vc}
\newcommand{\updated}{\begin{flushright} \footnotesize \color{gray} Updated \VCDateRAW \end{flushright}}

\begin{document}
\updated

\begin{minipage}{0.4 \textwidth}
\flushleft {\Huge \myauthor}
\end{minipage}
\begin{minipage}{0.6 \textwidth}
\flushright {\footnotesize \myaffiliation \\ \myaddress \\ \href{mailto:ncj@uchicago.edu}{\myemail} \\ \href{\myweb}{\myweb} \Large \textperiodcentered \footnotesize \href{http://www.twitter.com/nclarkjudd}{\mytwitter} }
\end{minipage}

\section{Education}

\di{Ph.D Candidate, Sociology, The University of Chicago}{}
\di{M.A. (in passing), The University of Chicago}{2015}
\di{B.A. (\emph{magna cum laude}), Metropolitan Studies (high honors) and Journalism and Mass Communication, New York University \newline --- Thesis: \emph{The Omniscient Policeman: Terror and Surveillance in the New York City Subways}.}{2007}

\section{Publications}

\hangindent=0.5in Messing, Solomon, Patrick VanKessel, Adam Hughes, Nick Judd, Rachel Blum and Brian Broderick. February 2017. ``Partisan Conflict and Congressional Outreach,'' Pew Research Center.

\section{Works in Progress}

\hangindent=0.5in Judd, Nick. ``Who Speaks in the People's House? Policy Talk and Political Inequality in the U.S. Congress.''

\hangindent=0.5in Martin, John Levi and Nick Judd. ``Ways of Viewing the Political: Theoretical Shifts.” Forthcoming in Thomas Janoski, Cedric de Leon, Joya Misra and Isaac Martin (eds.) \emph{The New Handbook of Political Sociology} (New York:	Cambridge University Press).

\section{Conference Papers and Invited Presentations (Since 2012)}
\subsection{``Who Speaks in the People's House? Policy Talk and Political Inequality in the U.S. Congress.''}

\di{American Sociological Association annual meeting, Philadelphia, Penn.}{August 2018}
\di{International Conference on Computational Social Science, Chicago, Ill.}{July 2018}
\di{Midwest Political Science Association annual meeting, Chicago, Ill.}{April 2018}

\subsection{``Hello From the Other Side: Interaction and Ideology in Elite Polarization.''}

\di{Pew Research Center, Washington, D.C.}{August 2016}

\subsection{``Who Controls the People's House? Agenda-Setting and Inequality in the U.S. Congress.''}

\di{American Sociological Association annual meeting, Seattle, Wa.}{August 2016}

\subsection{``Putting a ‘like’ to politics. Social network, digital democracy and new forms of politics and political communication'' (``Mettere un like alla politica. Social network, democrazia digitale e nuove forme di (comunicazione) politica'').}

\di{Invited speaker, Internet Festival, Scuola Normale Superiore, Pisa, Italy.}{October 2012}

\subsection{``Politics 2012 \& Social Media.''}

\di{Invited speaker, Social Media Weekend, Columbia University School of Journalism, New York, NY.}{January 2012}

\section{Other Activity (Workshops, on-campus conferences, etc.)}
\di{Conference co-organizer, Crossing Disciplinary Boundaries conference, The University of Chicago.}{September 2018}
\di{``Who Controls the People’s House? Policy Talk, Parliamentary Maneuvering, and Political Inequality in the U.S. Congress.'' Crossing Disciplinary Boundaries conference, The University of Chicago.}{September 2017}
\di{``Who Controls the People's House? Agenda-Setting and Inequality in the U.S. Congress.'' Politics, History and Society Workshop, The University of Chicago.}{May 2016}
\di{Discussant, Money, Markets and Governance Workshop, The University of Chicago.}{May 2016}
\di{Invited speaker, Macroanalysis in the Humanities, The University of Chicago.}{April 2016}
\di{Discussant, Politics, History and Society Workshop, The University of Chicago.}{April 2016}
\di{Computational Social Science Workshop, The University of Chicago.}{May 2015}
\di{Presider, ``Who Governs?'' Society for Social Research Spring Institute, University of Chicago, Chicago, Ill.}{May 2015}
\di{Conference co-organizer, 2015 Society for Social Research Spring Institute}{May 2015}
\di{Moderator, ``How to Cover the Data-Driven Campaign.'' Personal Democracy Forum, New York University, New York, NY.}{June 2013}
\di{Moderator, ``Do's and Don'ts for Civic Hackers.'' Personal Democracy Forum, New York University, New York, NY.}{June 2013}
\di{Moderator, ``Learning from the SENSEable city.'' Personal Democracy Forum, New York University, New York, NY.}{June  2012}

\section{Courses Taught}

\di{Lecturer, Social Science Inquiry I-III \\~\\ This is a three-quarter core sequence in the College at The University of Chicago. In this sequence, students gain a first exposure to empirical social science, with a focus on quantitative methods. The first quarter focuses on the motivation, intuition, and logic of empirical social science. The second quarter is an in-depth look at specifically quantitative methods, taught in the R statistical programming language. In the third quarter, students build on what they have learned by planning and executing their own research project.}{2018--2019 Academic Year}
\di{Lecturer, Social Science Inquiry II}{Winter 2017, Winter 2018}
\di{Lecturer, Mathematics for Social Science \\~\\ This is an intensive course for incoming first-year master's students in the social sciences. Taught in two lectures per day over the course of nine days, this course covers the mathematical foundations of quantitative social science, including topics in set theory, calculus, linear algebra, probability, and statistics. Offered Pass/Fail only.}{Autumn 2017}

\section{Other teaching experience}

\di{Teaching Assistant, Machine Learning and Policy}{Spring 2018}
\di{Teaching Assistant, Mathematics for Social Science}{Fall 2016}
\di{Teaching Intern, Social Science Inquiry III}{Spring 2016}
\di{Teaching Intern, Social Science Inquiry II}{Winter 2016}
\di{Teaching Intern, Social Science Inquiry I}{Fall 2015}

\section{Research Experience}
\di{Research Assistant\\Center for International Social Science Research, The University of Chicago}{2018}
\di{Research Assistant\\Justin Grimmer, Department of Political Science, The University of Chicago}{2018--}
\di{Consultant (working remotely through Ajilon)\\Pew Research Center, Washington, D.C.}{2016--2017}
\di{Data Labs Fellow\\Pew Research Center, Washington, D.C.}{2016}
\di{Research Assistant\\Center for an Urban Future, New York, N.Y.}{2006}

\section{Journalism Experience}

\di{Reporter and Editor\\techPresident.com, New York, N.Y.\\(2011-2013, managing editor; previously: associate editor, assistant editor, reporter/writer)}{2009-2013}
\di{Contributor\\Yahoo News}{2012}
\di{Staff Reporter\\\emph{The Riverdale Press}, New York, N.Y.}{2008-2009}
\di{Intern Reporter\\\emph{The Jersey Journal}, Jersey City, N.J.}{2007-2008}
\di{Editorial Assistant\\\emph{City Limits}, New York, N.Y.}{2007}

\section{Honors, Grants and Fellowships}
\di{Data Labs Fellowship, Pew Research Center, Washington, D.C.}{Summer 2016}
\di{Google Journalism Fellowship, Sunlight Foundation, Washington, D.C.}{Summer 2015}
\di{Social Science Research Fellowship, The University of Chicago, Chicago, Ill.}{2013--2018}
\di{Social science panel winner, Dean's Undergraduate Research Conference, New York University, New York, N.Y.}{April 2007}
\di{College of Arts and Science Presidential Honors Scholar, New York University, New York, N.Y.}{2005-2007}
\di{Sonoma County Press Club Scholarship, Sonoma County Press Club, Santa Rosa, Calif.}{2004}

\section{Service to Profession}

\di{Ad-hoc reviewer, \emph{American Journal of Sociology}.}{}


\section{Departmental Service}

\di{Representative, Society for Social Research}{2014--2015}

\section{In the Media}

Quoted or cited in Politico, \emph{The San Francisco Chronicle}, \emph{The Washington Post}, CNN.com, Theatlantic.com, The New York Times' Five Thirty Eight blog, and Poynter.com; live radio or webcast appearances for "Huffington Post Live,'' Variety, and WOSU. \\

As a working journalist, many news articles published for outlets including The Sunlight Foundation, Vice.com, Yahoo News, and others.

\section{Professional Memberships}

\di{American Sociological Association}{2013--}
\di{Society of Professional Journalists}{2007--2013}


\end{document}
%%% Local Variables: 
%%% coding: utf-8
%%% mode: latex
%%% TeX-engine: xetex
%%% End: